\def \testtasks {%
    \restartlist{enumerate}
    \begin{enumerate}


        \item Написать определение предела числовой последовательности.
        \item Может ли последовательность иметь два различных конечных предела?
        \item Написать в кванторах определение того, что последовательность имеет предел равный $+ \infty$.
        \item Сходится ли последовательность $x_n = 1$ ? Ответ обосновать.
        \item Сформулировать теорему Вейерштрасса.
        \item Определение числа е.
        \item Сформулируйте теорему о трёх последовательностях.
        \item Сформулируйте свойство предела частного двух числовых последовательностей.
        \item Пусть $x_n$ сходится к $a \in \mathbb{R}$ и $x_n > 0$ $\forall n \in \mathbb{N}$. Верно ли, что $a > 0$? Ответ обосновать.
        \item Привести пример двух последовательностей $x_n$ и $y_n$ таких, что $x_n$ и $x_n \cdot y_n$ сходятся, а $y_n$ -- расходится. 
    \end{enumerate}
}


\def \basetasks {%
    \restartlist{enumerate}
    \begin{enumerate}


        \item Доказать, что $x_n$ сходится к $a \in \mathbb{R}$  $\Leftrightarrow$ \\
        $1) \forall \varepsilon > 0 \exists N : \forall n \geqslant N \hookrightarrow |x_n - a| < C \varepsilon ; C > 0$\\
        $2) \forall \varepsilon > 0 \exists N : \forall n \geqslant N \hookrightarrow |x_n - a| \leq \varepsilon $.
        \item Доказать, что:\\
        1)если $\lim\limits_{n \to \infty} x_n = a$ $n \in \mathbb{N} x_n > 0$, то $\lim\limits_{n \to \infty} \sqrt{x_n} = \sqrt{a}$\\
        2)если $x_n \rightarrow a$, то $|x_n| \rightarrow |a|.$
        \item Доказать, что $\lim\limits_{n \to \infty} \dfrac{\sin n}{n} = 0$.
        \item Доказать, что последовательности $x_n = 2^n - 100n$ и $y_n = n \cdot \sin \dfrac{\pi n}{2}$ расходятся.
        \item Доказать, что если $x_n$ возрастает и неограничена, то у нее существует предел равный $+ \infty.$
        \item Доказать теорему о предельном переходе в неравенстве.
        \item Доказать, что:\\
    $1) \lim\limits_{n \to \infty} \dfrac{1}{n^p} = 0$, где $p \in \mathbb{N}$\\
    $2) \lim\limits_{n \to \infty} \dfrac{a^n}{n!} = 0$
        \item Найти предел последовательности: $1) x_n = \dfrac{3n + 5}{6n^2 +3n +8}$\\ $2) x_n = \sqrt{n^2 + 1} - \sqrt{n^2 -1}\\ 3) x_n = \dfrac{2n + \sin n + 100}{\sqrt{9n^2 + 10n +17}} .$     
        


    \end{enumerate}
}

