\def \def_deriv {%
    \item Напишите определение производной функции $f(x)$ в точке $x_0$. 
}

\def \def_right_deriv {
	\item Напишите определение \textit{правой} производной $f(x)$ в точке $x_0$.
}

\def \def_deriv_geom {
	\item В чем состоит геометрический смысл производной?
}
 
\def \def_diff {
	\item Дайте определение дифференцируемости функции $f(x)$ в точке $x_0$.
}       

\def \def_differential {
	\item Напишите определение дифференциала функции $f(x)$ в точке $x_0$.
}

\def \def_diff_segment {
	\item Дайте определение дифференцируемости функции $f(x)$ на отрезке $[a, b]$.
}

\def \def_second_deriv {
	\item Дайте определение второй производной функции $f(x)$ в точке $x_0$. 
}

\def \def_n_deriv {
	\item Дайте определение производной порядка n функции $f(x)$ в точке $x_0$. 
}

\def \def_second_differential {
	\item Напишите определение второго дифференциала функции $f(x)$ в точке $x_0$.   
}

\def \def_n_differential {
	\item Напишите определение дифференциала порядка n функции $f(x)$ в точке $x_0$.
}

\def \def_Taylor_polinom {
	\item Напишите определение многочлена Тейлора функции $f(x)$ в точке $x_0$. В чем состоит основное свойство многочлена Тейлора?
}

\def \def_loc_min {
	\item Дайте определение локального минимума функции $f(x)$. 
}

\def \def_loc_max {
	\item Дайте определение локального максимума функции $f(x)$. 
}

\def \def_convex {
	\item Дайте определение выпуклой вниз (вверх) функции $f(x)$ на интервале $(a, b)$. 	
}

\def \def_inflection {
	\item Дайте определение точки перегиба функции $f(x)$.
}

\def \def_asymp {
	\item Напишите определение невертикальной асимптоты графика функции $f(x)$.
}

\def \def_vert_asymp {
	\item Напишите определение вертикальной асимптоты графика функции $f(x)$. 
}

\def \def_vect_func {
	\item Напишите определение вектор-функции $\vec{r}(t)$.
} 

\def \def_lim_vect_func {
	\item Напишите определение предела вектор-функции $\vec{r}(t)$ в точке $t_0$.
} 

\def \def_curve {
	\item Напишите определение непрерывной пространственной кривой.
}

\def \def_simple_curve {
	\item Дайте определение простой кривой.
}

\def \def_contour {
	\item Напишите определение контура. Какая кривая называется \textit{простым} контуром?
}

\def \def_cd_curve {
	\item Какая кривая называется непрерывно дифференцируемой?
}

\def \def_smooth_curve {
	\item Какая кривая называется гладкой? Что такое кусочно-гладкая кривая?
}

\def \def_perm_change {
	\item Дайте определение допустимой замены параметра на гладкой кривой. Как определить допустимую замену параметра на гладкой \textit{ориентированной} кривой?
}

\def \def_tangent {
	\item Напишите определение касательной к кривой $\Gamma$ в точке $\vec{r} (t_0)$. 
}

\def \def_length_polyline {
	\item Напишите определение длины ломаной, вписанной в кривую $\Gamma$. 
}

\def \def_length_curve {
	\item Напишите определение длины кривой $\Gamma$. Что такое спрямляемая кривая?
}

\def \def_arc_length {
	\item Дайте определение переменной длины дуги $s(t)$ кривой $\Gamma$. 
}

\def \def_norm {
	\item Дайте определение нормали к пространственной кривой $\Gamma$ в точке $\vec{r} (t_0)$. Как определяется главная нормаль к кривой в этой точке?
}

\def \def_curvature {
	\item Дайте определение кривизны кривой $\Gamma$ в точке $\vec{r} (t_0)$. Что такое радиус кривизны кривой? 
}

\def \def_binorm {
	\item Как определяется единичный вектор бинормали к кривой $\Gamma$ в точке $\vec{r} (t_0)$?
}

\def \def_planes {
	\item Напишите уравнения нормальной, спрямляющей и соприкасающейся плоскостей.
}

\def \def_antider {
	\item Напишите определение первообразной функции $f(x)$ на интервале $(a, b)$.
} 

\def \def_int {
	\item Что называется неопределенным интегралом от функции $f(x)$?
}

\def \form_deriv_summ {
	\item Сформулируйте правило вычисления производной суммы.
}

\def \form_deriv_mul {
	\item Сформулируйте правило вычисления производной произведения.
}

\def \form_deriv_div {
	\item Сформулируйте правило вычисления производной частного.
}

\def \form_deriv_inverse {
	\item Сформулируйте теорему о производной обратной функции.
}

\def \form_diff_complex {
	\item Сформулируйте теорему о дифференцируемости сложной функции.
}

\def \form_differential {
	\item Напишите формулу, задающую связь между дифференциалом и производной функции $f(x)$ в точке $x_0$.
}

\def \form_linear_n_deriv {
	\item Сформулируйте свойство линейности n-й производной.
}

\def \form_Leybnitz {
	\item Сформулируйте теорему о формуле Лейбница.
}

\def \form_n_differential {
	\item Напишите формулу, задающую связь между  n-й производной и n-м дифференциалом функции $f(x)$ в точке $x_0$. 
}

\def \form_Ferma {
	\item Сформулируйте теорему Ферма.
}

\def \form_Roll {
	\item Сформулируйте теорему Ролля.
}

\def \form_Lagrange {
	\item Сформулируйте теорему Лагранжа о среднем.
}

\def \form_Koshi {
	\item Сформулируйте теорему Коши о среднем.
}

\def \form_Taylor_Peano {
	\item Сформулируйте теорему о формуле Тейлора с остаточным членом в форме Пеано.
} 

\def \form_Taylor_Lagrange {
	\item Сформулируйте теорему о формуле Тейлора с остаточным членом в форме Лагранжа.
}

\def \form_Taylor_diff {
	\item Сформулируйте теорему о дифференцировании формулы Тейлора.
}

\def \form_Lopital_0 {
	\item Сформулируйте правило Лопиталя для неопределенности вида $\dfrac{0}{0}$.
}

\def \form_Lopital_inf {
	\item Сформулируйте правило Лопиталя для неопределенности вида $\dfrac{\infty}{\infty}$.
}

\def \form_crit_mon {
	\item Сформулируйте критерий монотонности для дифференцируемой на интервале $(a, b)$ функции $f(x)$.
}

\def \form_suff_mon {
	\item Сформулируйте достаточное условие строгой монотонности для дифференцируемой на интервале $(a, b)$ функции $f(x)$. 
}

\def \form_suff_extr_1 {
	\item Сформулируйте первое достаточное условие экстремума.
}

\def \form_suff_extr_2 {
	\item Сформулируйте второе достаточное условие экстремума.
}

\def \form_suff_extr_3 {
	\item Сформулируйте третье достаточное условие экстремума.
}

\def \form_crit_convex {
	\item Сформулируйте критерий выпуклости для дважды дифференцируемой на интервале $(a, b)$ функции $f(x)$.
}

\def \form_lim_vect_func {
	\item Напишите формулировку теоремы о связи предела вектор-функции с пределами компонент.
}

\def \form_Lagrange_vect {
	\item Напишите формулировку теоремы Лагранжа о среднем для вектор-функций. 
}

\def \form_length_arc {
	\item Напишите формулу для производной переменной длины дуги $s'(t)$ на непрерывно дифференцируемой кривой, заданной вектор-функцией $\vec{r} (t)$.
}

\def \form_curvature {
	\item Напишите формулу для расчета кривизны гладкой кривой, заданной вектор-функцией $\vec{r} (t)$.
}

\def \form_antider {
	\item Напишите формулировку теоремы о множестве первообразных заданной на интервале $(a, b)$ функции $f(x)$.
} 

\def \form_int_linear {
	\item Напишите свойство линейности неопределенного интеграла.
}

\def \form_int_part {
	\item Напишите формулу для интегрирования функции по частям. 
}

\def \proof_cont_deriv {
	\item Докажите, что функция, имеющая в точке $x_0$ конечную производную, непрерывна в этой точке. 
}

\def \proof_deriv_mul {
	\item Докажите правило вычисления производной произведения.
}

\def \proof_deriv_div {
	\item Докажите правило вычисления производной частного.
}

\def \proof_deriv_inverse {
	\item Докажите теорему о производной обратной функции.
}

\def \proof_deriv_eq_diff {
	Докажите, что функция дифференцируема в точке тогда и только тогда, когда имеет в этой точке конечную производную.
}

\def \proof_diff_complex {
	\item Докажите теорему о дифференцируемости сложной функции.
	\begin{enumerate}
		\item[1)] Сформулируйте теорему.
		\item[2)] В условиях теоремы докажите, что сложная функция определена в некоторой окрестности точки $x_0$.
		\item[3)] Получите определение того, что сложная функция является дифференцируемой в точке $x_0$, используя определения дифференцируемости функций $f(x)$ в точке $x_0$ и $g(y)$ в точке $y_0$.
	\end{enumerate}
}

\def \proof_inv_differential {
	\item Покажите, что форма первого дифференциала функции $f(x)$ в точке $x_0$, дифференцируемой в этой точке, инвариантна относительно замены переменной.
}

\def \proof_linear_n_deriv {
	\item Докажите свойство линейности производной порядка n.
}

\def \proof_Leybnitz {
	\item Докажите теорему о формуле Лейбница, используя метод математической индукции.
	\begin{enumerate}
		\item[1)] Сформулируйте теорему.
		\item[2)] Покажите, что выполнена база индукции.
		\item[3)] Покажите, что выполнен шаг индукции.
	\end{enumerate}
}

\def \proof_not_inv_differential {
	\item Покажите, что форма второго дифференциала функции $f(x)$ в точке $x_0$ не является, вообще говоря, инвариантной относительно замены переменной. 
}

\def \proof_n_differential {
	\item Получите формулу, задающую связь между  n-й производной и n-м дифференциалом функции $f(x)$ в точке $x_0$. 
}

\def \proof_Ferma {
	\item Докажите теорему Ферма.
}

\def \proof_Roll {
	\item Докажите теорему Ролля.
}

\def \proof_Lagrange {
	\item Докажите теорему Лагранжа о среднем.
}

\def \proof_Koshi {
	\item Докажите теорему Коши о среднем.
}

\def \proof_Taylor_Peano {
	\item Докажите теорему о формуле Тейлора с остаточным членом в форме Пеано.
	\begin{enumerate}
		\item[1)] Сформулируйте теорему.
		\item[2)] Получите выражение для отношения $\dfrac{r_n (x)}{\psi (x)}$, где $\psi (x) = (x - x_0)^n$.
		\item[3)] Докажите, что $\lim\limits_{x \to x_0} \dfrac{r_n (x)}{\psi (x)} = 0$, используя теорему о замене переменной под знаком предела и определение производной.
	\end{enumerate}
} 

\def \proof_Taylor_Lagrange {
	\item Докажите теорему о формуле Тейлора с остаточным членом в форме Лагранжа.
	\begin{enumerate}
		\item[1)] Сформулируйте теорему.
		\item[2)] Получите выражение для $f^{n + 1} (x)$ через производную высшего порядка остаточного члена $r_n (x)$.
		\item[3)] Получите выражение для остаточного члена $r_n (x)$, последовательно применяя теорему Коши о среднем.
	\end{enumerate}
}

\def \proof_Taylor_diff {
	\item Докажите теорему о дифференцировании формулы Тейлора.
}

\def \proof_Taylor_uniq {
	\item Докажите теорему о единственности разложения по формуле Тейлора.
}

\def \proof_Lopital_0 {
	\item Докажите правило Лопиталя для неопределенности вида $\dfrac{0}{0}$.
}

\def \proof_crit_mon {
	\item Докажите критерий монотонности для дифференцируемой на интервале $(a, b)$ функции $f(x)$.
}

\def \proof_suff_mon {
	\item Докажите достаточное условие строгой монотонности для дифференцируемой на интервале $(a, b)$ функции $f(x)$. 
}

\def \proof_suff_extr_1 {
	\item Докажите первое достаточное условие экстремума.
}

\def \proof_suff_extr_3 {
	\item Докажите третье достаточное условие экстремума.
}

\def \proof_crit_convex {
	\item Пусть $f(x)$ дважды дифференцируема на интервале $(a, b)$. Докажите, что если $f''(x) \geq 0$ на $(a, b)$, то функция $f(x)$ - выпукла вниз на $(a, b)$.  
}

\def \proof_lim_vect_func {
	\item Докажите теорему о связи предела вектор-функции с пределами компонент.
}

\def \proof_Lagrange_vect {
	\item Докажите теорему Лагранжа о среднем для вектор-функций. 
}

\def \proof_length_curve {
	\item Докажите теорему о спрямляемости непрерывно дифференцируемой кривой.
}

\def \proof_curvature {
	\item Выведите формулу для расчета кривизны гладкой кривой, заданной вектор-функцией $\vec{r} (t)$.
}

\def \proof_antider {
	\item Докажите теорему о множестве первообразных заданной на интервале $(a, b)$ функции $f(x)$.
} 

\def \proof_int_linear {
	\item Докажите свойство линейности неопределенного интеграла.
}

\def \problem_sup {
	\item Приведите пример счетного числового множества, точная нижняя грань которого равна $-\infty$.
}

\def \problem_lim_sequence {
	\item Докажите, что $\lim\limits_{n\to \infty} x_n = a \in \mathbb{R}$ тогда и только тогда, когда $\forall \epsilon > 0 \exists N: \forall n > N |x_n - a| < C \epsilon$, где $C > 0$ - положительная константа.
}

\def \problem_lim_arifm {
	\item Приведите пример расходящихся последовательностей $x_n$ и $y_n$ таких, что $x_n \cdot y_n$ и $\dfrac{x_n}{y_n}$ - сходящиеся последовательности. 
}

\def \problem_subsequence {
	\item Докажите, что у неограниченной сверху последовательности существует подпоследовательность, которая стремится к $+\infty$.
} 

\def \problem_count {
	\item Докажите, что множество $\mathbb{N}_0 = \mathbb{N} \cup {0}$ является счетным.
}

\def \problem_fund {
    \item Докажите, что у фундаментальной последовательности любая ее подпоследовательность фундаментальна.       
}

\def \problem_partial_lim {
	\item Приведите пример последовательности, у которой ровно $p$ различных частичных пределов, $p \in \mathbb{N}$.
}

\def \problem_lim_sign {
	\item Пусть $\lim\limits_{x \to x_0} f(x) = A > 0$. Докажите, что $ \exists \delta > 0: f(x) > 0 \forall x \in \stackrel{o}{U}_{\delta} (x_0)$. 
}

\def \problem_cont_complex {
	\item Пусть функция $f(x)$ непрерывна в $x_0$. Верно ли, что $|f(x)|$ - непрерывная в точке $x_0$ функция? Ответ обоснуйте.   
} 

\def \problem_cont_segment {
	\item Пусть $f(x)$ определена на $[a, b]$, при этом множество значений этой функции - отрезок (иными словами, $f(x)$ переводит отрезок в отрезок). Cледует ли отсюда, что $f(x)$ непрерывна на $[a, b]$? Ответ обоснуйте.
}

\def \problem_cont_int {
	\item Пусть $f(x)$ непрерывна на интервале $(0, 1)$ и $f(0 + 0) = 1$. Кроме того, известно, что $f(x) \neq 0$  $(0, 1)$. Докажите, что $f(x) > 0$ на $(0, 1)$. 
}

\def \problem_trig {
	\item Докажите неравенство $\dfrac{\sin x}{x} - \cos x > 0 \forall x \in (0, \pi)$.
}

\def \problem_equiv {
	\item Докажите, что $f \sim g$ при $x \to x_0$ тогда и только тогда, когда $f = g + o(g)$ при $x \to x_0$.
}

\def \problem_max_deriv {
	\item Найдите максимальный порядок производной функции $f(x) = x^n |x|$ в нуле, $n \in \mathbb{N}$.
}
