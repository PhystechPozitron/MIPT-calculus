\def \testtasks {%
    \restartlist{enumerate}
    \begin{enumerate}


        \item Сформулируйте определение подпоследовательности.
        \item Найти все частичные пределы последовательности $x_n = (-1)^n$.
        \item Напишите определение верхнего предела последовательности.
        \item Сформулируйте теорему Больцано-Вейерштрасса.
        \item Приведите пример последовательности, не имеющей конечного частичного                  предела.
        \item Может ли сходящаяся последовательность иметь более одного частичного                  предела.
        \item Напишите определение того, что последовательность является                            фундаментальной.
        \item Может ли фундаментальная последовательность быть неограниченной?
        \item Сформулируйте критерий Коши для числовых последовательностей.
        \item Приведите пример последовательности, которая имеет предел, но не является         фундаментальной.  
    \end{enumerate}
}


\def \basetasks {%
    \restartlist{enumerate}
    \begin{enumerate}


        \item Доказать, что $\lim\limits_{n \to \infty} \dfrac{n^3 + b}{n^3} = 1$ и $               \lim\limits_{n \to \infty} \dfrac{n}{n^2 + 10} = 0 $ по определению.
        \item Доказать, что последовательность $x_n = \dfrac{n^2 - 10}{n}$ расходится.
        \item Найти все частичные пределы последовательностей и доказать, что других                частичных пределов нет: \\
        $1) x_n = \cos \dfrac{\pi n}{4};$ $2) x_n = (-1)^n \cdot \cos \dfrac{\pi n}{4};$\\
        \item Найти все частичные пределы последовательностей:\\
        $1) x_n = \dfrac{(-1)^n}{n}$\\
        $2) y_n= \begin{cases}  
                    n, n = 2k +1 \\
                    0, n = 2k \\
                  \end{cases} $ \\
        $3) z_n = n^{(-1)^n}$\\
        
        \item Для каждой последовательности выяснить, является ли она фундаментальной,              сходящейся, ограниченной (по определению):\\
        $1) x_n = \dfrac{(-1)^n}{n};$ $2) y_n = n ; $ $3) x_n = (-1)^nn .$
        \item Используя критерий Коши, исследовать последовательности на сходимость:\\
        $1) x_n = \sin \dfrac{\pi n}{2};$
        $2) y_n = \sum \limits_{k = 1}^{n} \dfrac{\cos k}{2^k}$
        \item Пусть $x_n$ сходится к $a \in \mathbb{R}$. Доказать, что $a$ единственный         частичный предел $x_n$.
        \item Доказать, что у возрастающей и неограниченной сверху последовательности               существует частичный предел, равный $+ \infty$.
        \item Пусть $x_n$ и $y_n$ фундаментальны. Доказать, что $x_n \cdot y_n$                     фундаментальна.
        

       
        


    \end{enumerate}
}

