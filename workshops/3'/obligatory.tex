\def \testtasks {%
    \restartlist{enumerate}
    \begin{enumerate}
        \item Сформулируйте теорему Кантора о вложенных отрезках.
        \item Может ли последовательность вложенных отрезков иметь более одной общей точки? 
        \item Сформулируйте определение подпоследовательности.
        \item Напишите определение верхнего предела последовательности.
        \item Сформулируйте теорему Больцано-Вейерштрасса.
        \item Приведите пример последовательности, не имеющей конечного частичного предела.
        \item Напишите определение того, что последовательность является фундаментальной.
        \item Может ли фундаментальная последовательность быть неограниченной?
        \item Сформулируйте критерий Коши для числовых последовательностей.
        \item Приведите пример последовательности, которая имеет предел, но не является         фундаментальной.  
    \end{enumerate}
}


\def \basetasks {%
    \restartlist{enumerate}
    \begin{enumerate}

        \item Доказать, что $\lim\limits_{n \to \infty} \dfrac{n^3 + b}{n^3} = 1$ и $\lim\limits_{n \to \infty} \dfrac{n}{n^2 + 10} = 0 $ по определению.

        \item Доказать, что последовательность $x_n = \dfrac{n^2 - 10}{n}$ расходится.
        
        \item Найти все частичные пределы последовательностей:\\
        $1) x_n = \dfrac{(-1)^n}{n}$\\
        $2) x_n= \begin{cases}  
                    n, n = 2k +1 \\
                    0, n = 2k \\
                  \end{cases} $ \\
        $3) x_n = n^{(-1)^n}$\\
        $4) x_n = (-1)^n \cdot \cos \dfrac{\pi n}{4};$ 

        \item Используя критерий Коши, исследовать последовательности на сходимость:\\
        $1) x_n = n^2 \cdot \sin \dfrac{\pi n}{2};$
        $2) x_n = \sum \limits_{k = 1}^{n} \dfrac{\cos k}{2^k}$
        $3) x_n = n^{(-1)^n} $

        \item Пусть $x_n$ сходится к $a \in \mathbb{R}$. Доказать, что $a$ -  единственный частичный предел $x_n$.

        \item Доказать, что у неограниченной сверху последовательности существует частичный предел, равный $+ \infty$. Пользуясь этим фактом, сформулировать обобщение теоремы Больцано-Вейерштрасса на произвольные последовательности.

        \item Пусть $x_n$ и $y_n$ фундаментальны. Доказать, что $x_n \cdot y_n$ фундаментальна.

    \end{enumerate}
}

