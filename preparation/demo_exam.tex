\documentclass[a4paper,12pt]{article}

\usepackage{multirow}
\usepackage[left=1cm,right=1cm,
top=2cm,bottom=2cm,bindingoffset=0cm]{geometry}
\usepackage[utf8]{inputenc}
\usepackage[russian]{babel}
\usepackage[T2A]{fontenc}
\usepackage{amsfonts,longtable, amssymb, amsmath}
\usepackage{graphicx}
\usepackage{wrapfig}
\usepackage{hyperref}
%\urlstyle{same}
\usepackage{pdfpages}
%\usepackage{pscyr}
\usepackage[normalem]{ulem}  % для зачекивания текста
\usepackage{ulem}
\graphicspath{{pictures/}}
\DeclareGraphicsExtensions{.pdf,.png,.jpg, .svg}
\usepackage{pgfplots}
\pgfplotsset{compat=1.12}
\usepackage{fancyhdr}
\pagestyle{fancy}
\fancyhead{}
\fancyhead[LO]{} 
\fancyhead[CO]{}
\fancyhead[RO]{}

\usepackage{graphicx,url}

\usepackage[russian]{babel}   
%\usepackage[latin1]{inputenc}  
\usepackage[utf8]{inputenc}  
\usepackage{verbatim}
\begin{document}

\section*{Демо-версия экзамена по практике.}

\begin{enumerate}
     	\item Вычислите интегралы:
     	\begin{enumerate}
     		\item[а)] $ \displaystyle \int \dfrac{2x^4 - 2x^3 - x^2 + 2}{(x - 1)(2x^2 - 2x + 1)} \ dx$;
     		\item[б)] $ \displaystyle \int \dfrac{\arcsin \sqrt{x}}{(1 - x)^{3/2}} \ dx$.
     	\end{enumerate} 
     	
     	\item Найдите производную порядка $n$, $n > 1$, функции \[ y(x) = x^2 \ln \sqrt[3]{(2x + 3)^2}. \]
     	\item Разложите функцию $f(x) = \bigg( \dfrac{x^2}{2} + x \bigg) \sqrt{1 - 3x}$ по формуле Тейлора в окрестности точки $x_0 = -1$ до $o( (x+1)^n )$.  
     	\item Вычислите предел \[ \lim\limits_{x \to 0} \dfrac{\cos (x \sqrt{1 + x}) + \ln (1 + x + x^2) - \arcsin x - 1}{e^{\tg x} - \ch x - x}. \]
     	\item Вычислите предел \[ \lim\limits_{x \to 0} \bigg( \dfrac{e^{x \sqrt{1 + 2x}} - \cos(x - x^2) - 2x^2}{\arcsin x} \bigg)^{\ctg^2 x}. \] 
     	\item Постройте график функции \[ y(x) = \dfrac{3 (x - 2)^2 - (x - 1)^3}{(x - 2)^2} .\]  
     	\item Найдите наибольшую кривизну кривой $y(x) = 2 \ln \bigg( 1 - \dfrac{x^2}{4} \bigg)$. 
     	\item Докажите, что последовательность сходится, и найдите её предел:
     		\[ x_1 = 0, \ x_{n+1} = \dfrac{1}{2} \sqrt{25 + 15 x_n}. \] 	 	     	
\end{enumerate}

\end{document}