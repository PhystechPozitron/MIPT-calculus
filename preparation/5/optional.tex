\def \Dirihle {%
    \item Показать, что функция Дирихле $ f(x) = \begin{cases}
	1, \; x \in \mathbb{I}\\
	0, \; x \in \mathbb{Q} \\
	\end{cases}$ разрывна в каждой точке $x_0 \in \mathbb{R}$.
}

\def \FindFInXzero {%
    \item Пусть $f(x)$ непрерывна в $x_0$ и $\exists \delta > 0 : 
	\begin{cases}
		\forall x \in \stackrel{o}{U}_{\delta}(x_0 + 0) \hookrightarrow f(x) > 0 \\
		\forall x \in \stackrel{o}{U}_{\delta}(x_0 - 0) \hookrightarrow f(x) < 0  \\
	\end{cases} $. Найти $f(x_0)$.
}

\def \ProveContOnSegment {%
    \item Пусть $f(x)$ непрерывна на отрезках $[a,b]$ и $[b,c]$. Доказать, что $f(x)$ непрерывна на $[a,c]$.
}

\def \FindLimits {%
    \item Найти пределы:\\
	$1)\; \lim\limits_{x \rightarrow 0} \dfrac{\arctan x}{x}$ \; $2)\; \lim\limits_{x \rightarrow \pi } \dfrac{1 + \cos x}{(x - \pi)^2}$ \; \textit{ (Указание: $\lim\limits_{x \rightarrow 0} \dfrac{\sin x}{x} = 1$)} 
}

\def \ContinAandB {%
    \item Пусть $\Psi(x) = \begin{cases}
	A\cos x , \; |x| < a \\	
	B\cdot e^{-|x|}, \; |x| > a \\
	\end{cases}$. При каком соотношении между A и B функция $\Psi(x)$ непрерывна на $\mathbb{R}$. 
}
\def \UnContinMonoton {%
    \item Пусть $f(x)$ монотонна на $(a,b)$ и испытывает разрыв в точке $c = \dfrac{a + b}{2}$. Доказать, что точка $c$ не может быть точкой разрыва второго рода.
}
\def \ProveDirihleOnXContin {%
    \item Доказать, что функция $f(x) = \begin{cases}
	x , \; x \in \mathbb{I} \\
	0, \; x \in \mathbb{Q}\\ 
	\end{cases}$ непрерывна в нуле.
}
\def \FindExampleDirihleOnX {%
    \item Привести пример функции, которая является непрерывной в нуле и разрывной в каждой точке, отличной от нуля.
}
\def \FindUncontinWithType {%
    \item Найти все точки разрыва функции $f(x) = \dfrac{\sin e^{\frac{1}{x}}}{x - 1}$ и указать их тип.
}
\def \ProveContinAndLimitation {%
    \item Доказать, что если функция непрерывна в точке, то она ограниченна в некоторой окрестности этой точки.
}
\def \FindExample {%
    \item Привести пример разрывной функции, определенной на отрезке и имеющей в качестве множества значений отрезок.
}
\def \FindExamoleCountable {%
    \item Привести пример функции, которая имеет счетное число точек разрыва.
}
\def \Geometry {%
    \item На плоскости задан многоугольник М и прямая L. Доказать, что существует прямая $L'$, паралельная $L$ и делящая заданный многоугольник на две равные части.
}
