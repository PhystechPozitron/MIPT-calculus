\documentclass[a4paper,12pt]{article}

\usepackage{multirow}
\usepackage[left=1cm,right=1cm,
top=2cm,bottom=2cm,bindingoffset=0cm]{geometry}
\usepackage[utf8]{inputenc}
\usepackage[russian]{babel}
\usepackage[T2A]{fontenc}
\usepackage{amsfonts,longtable, amssymb, amsmath}
\usepackage{graphicx}
\usepackage{wrapfig}
\usepackage{hyperref}
%\urlstyle{same}
\usepackage{pdfpages}
%\usepackage{pscyr}
\usepackage[normalem]{ulem}  % для зачекивания текста
\usepackage{ulem}
\graphicspath{{pictures/}}
\DeclareGraphicsExtensions{.pdf,.png,.jpg, .svg}
\usepackage{pgfplots}
\pgfplotsset{compat=1.12}
\usepackage{fancyhdr}
\pagestyle{fancy}
\fancyhead{}
\fancyhead[LO]{} 
\fancyhead[CO]{}
\fancyhead[RO]{}

\usepackage{graphicx,url}

\usepackage[russian]{babel}   
%\usepackage[latin1]{inputenc}  
\usepackage[utf8]{inputenc}  
\usepackage{verbatim}
\begin{document}

\section*{Подготовка к практике, день 4}

\begin{enumerate}
     	\item Вычислите интегралы от рациональных функций:	
     	\begin{enumerate}
     		\item[1)] $\displaystyle \int \dfrac{x^3 + 2x^2 + 2}{(x^2 - x + 1)(x^2 + x + 1)} dx$ \textit{(2003 - 2004, вариант 1)};
     		\item[2)] $\displaystyle \int \dfrac{x^3 + x}{(x + 1)(x^2 + 2x + 2)} dx$ \textit{(2004 - 2005, вариант 1)};
     		\item[3)] $\displaystyle \int \dfrac{-2x^3 + 4x^2 -3x - 2}{(x + 2)(2x^2 -x + 2)} dx$ \textit{(2007 - 2008, вариант 1)}.
     	\end{enumerate} 
     	
     	\item Вычислите интегралы от трансцендентных функций:	
     	\begin{enumerate}
     		\item[1)] $\displaystyle \int (x^3 + x) \cos{x^2} dx$ \textit{(1997 - 1998, вариант 1)};
     		\item[2)] $\displaystyle \int \dfrac{(\arccos \ln x)^2}{x} dx$ \textit{(2002 - 2003, вариант 1)};
     		\item[3)] $\displaystyle \int \dfrac{1 + \tg^3 x}{1 + \sin(2x)} dx$ \textit{(2002 - 2003, вариант 1)}.
     		\item[4)] $\displaystyle \int (x^2 + 1) \sqrt{16 - x^2} \ dx$ \textit{(2007 - 2008, вариант 1)}. 
     	\end{enumerate}
     	     	
\end{enumerate}

\textit{Замечание.} Не забывайте константу в ответе! 

\end{document}