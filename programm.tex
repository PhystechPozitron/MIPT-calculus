\documentclass[a4paper,12pt]{article}

\usepackage{multirow}
\usepackage[left=1cm,right=1cm,
top=2cm,bottom=2cm,bindingoffset=0cm]{geometry}
\usepackage[utf8]{inputenc}
\usepackage[russian]{babel}
\usepackage[T2A]{fontenc}
\usepackage{amsfonts,longtable, amssymb, amsmath}
\usepackage{graphicx}
\usepackage{wrapfig}
\usepackage{hyperref}
%\urlstyle{same}
\usepackage{pdfpages}
%\usepackage{pscyr}
\usepackage[normalem]{ulem}  % для зачекивания текста
\usepackage{ulem}
\graphicspath{{pictures/}}
\DeclareGraphicsExtensions{.pdf,.png,.jpg, .svg}
\usepackage{pgfplots}
\pgfplotsset{compat=1.12}
\usepackage{fancyhdr}
\pagestyle{fancy}
\fancyhead{}
\fancyhead[LO]{} 
\fancyhead[CO]{}
\fancyhead[RO]{}

\usepackage{graphicx,url}

\usepackage[russian]{babel}   
%\usepackage[latin1]{inputenc}  
\usepackage[utf8]{inputenc}  
\usepackage{verbatim}
\begin{document}

\begin{center}
	\section*{Программа экзамена по теории (матанализ).}
	\textbf{Первый модуль.}
\end{center}

\begin{enumerate}

	\item Определения точной верхней и точной нижней граней числового множества. Теорема о существовании и единственности супремума. Определение счетности числового множества. Теорема о счетности множества рациональных чисел.
	\item Определение числовой последовательности. Свойства последовательностей - монотонность и ограниченность. Определения предела последовательности и сходящейся последовательности. Теоремы о единственности предела и ограниченности сходящейся последовательности. Теорема Вейерштрасса о пределе монотонной ограниченной последовательности. Определение числа e, обоснование существования предела из определения числа e \textit{(без доказательства монотонности)}.
	\item Свойства предела последовательности, связанные с неравенствами - переход к пределу в неравенстве и теорема о двух милиционерах. Определение бесконечно малой последовательности, арифметические свойства бесконечно малых последовательностей. Арифметические свойства предела последовательности.  
	\item Определения последовательности вложенных отрезков и стягивающейся последовательности вложенных отрезков. Теорема Кантора о вложенных отрезках. Теорема о несчетности множества действительных чисел.
	\item Определения подпоследовательности, частичного предела, верхнего и нижнего пределов последовательности. Теорема Больцано-Вейерштрасса. 
	\item Определение фундаментальноcти последовательности, ограниченность фундаментальной последовательности. Критерий Коши сходимости числовой последовательности.
	\item Определение числовой функции. Определения монотонной, ограниченной, периодической функций. Определение предела функции по Коши. Определения последовательности Гейне и предела функции по Гейне. Эквивалентность определений предела по Коши и по Гейне \textit{(без доказательства)}. Определение односторонних пределов. Свойства предела функции, связанные с неравенствами. Арифметические свойства предела. Теорема о существовании односторонних пределов у монотонных функций. Критерий Коши для функций \textit{(без доказательства)}.
	\item Определения непрерывности функции в точке, непрерывности справа и слева. Классификация точек разрыва. Определение сложной функции (композиции). Теорема о непрерывности композиции. Формулировка теоремы о пределе композиции.
	\item Определение непрерывности функции на интервале, полуинтервале и отрезке. Теорема Вейерштрасса о непрерывной на отрезке функции. Теорема Больцано-Коши о промежуточных значениях непрерывной функции. Следствие из теорем Вейерштрасса и Больцано-Коши о переводе непрерывной функцией отрезка в отрезок \textit{(без доказательства)}.
	\item Определение обратной функции. Теорема об обратной функции.
	\item Тригонометрическое неравенство. Доказательство первого замечательного предела. Доказательство непрерывности синуса, косинуса и тангенса. Определение и обоснование непрерывности функций, обратных к тригонометрическим. Определение степенных функций с целым, рациональным и действительным показателями. Доказательство второго замечательного предела.
	\item Определение эквивалентности функций в точке. Определение о-малого.       
	
\end{enumerate}

\newpage

\begin{center}
	\textbf{Второй модуль.}
\end{center}	 

\begin{enumerate}

	\item Определения производной функции в точке, односторонних производных. Геометрический смысл производной. Непрерывность функции, имеющей производную. Арифметические свойства производных. Теорема о производной обратной функции.
	\item Определение дифференцируемости функции в точке. Эквивалентность дифференцируемости и существования производной в точке. Теорема о дифференцируемости сложной функции. Определение дифференцируемости функции на интервале, полуинтервале и отрезке. Определение дифференциала функции в точке, формула связи дифференциала и производной. Инвариантность формы дифференциала относительно замены переменной.
	\item Определение второй производной функции в точке. Определение производной порядка n. Свойство линейности производной порядка n. Теорема о формуле Лейбница. Определение второго дифференциала функции в точке, неинвариантность формы второго дифференциала относительно замены переменной. Определение дифференциала порядка n. Формула связи дифференциала и производной порядка n.
	\item Теоремы о среднем Ферма, Ролля, Лагранжа и Коши.
	\item Определение и основное свойство многочлена Тейлора. Теорема о формуле Тейлора с остаточным членом в форме Пеано. Теоремы о единственности разложения функции по формуле Тейлора и о дифференцировании формулы Тейлора. Теорема о формуле Тейлора с остаточным членом в форме Лагранжа.     
	\item Теоремы о правиле Лопиталя для неопределенности вида $\dfrac{0}{0}$ и $\dfrac{\infty}{\infty}$ \textit{(случай $\dfrac{\infty}{\infty}$ - без доказательства)}.
	\item Критерий монотонности функции, достаточное условие строгой монотонности. Определение локального экстремума функции. Первое, второе и третье достаточные условия экстремума. Определения выпуклой вверх и выпуклой вниз функции. Критерий выпуклости \textit{(без доказательства необходимости)}. Определение точки перегиба функции. Определение вертикальной и невертикальной асимптот графика функции.
	\item Определение вектор-функции. Определение предела вектор-функции, теорема о связи предела вектор-функции с пределами компонент. Теорема Лагранжа о среднем для вектор-функций.
	\item Определение непрерывной кривой. Классификация кривых. Определение допустимой замены параметра на гладкой кривой. Определение допустимой замены параметра на гладкой \textit{ориентированной} кривой. 
	\item Определение касательной к кривой. Уравнение касательной, записанное через производную вектор-функции, задающей кривую. Определение длины \textit{ломаной}, вписанной в кривую. Определения длины кривой и спрямляемости. Теорема о спрямляемости непрерывно дифференцируемой кривой. Определение переменной длины дуги кривой. Формула для производной переменной длины дуги \textit{без доказательства}.
	\item Определения нормали и главной нормали к пространственной кривой. Определения кривизны и радиуса кривизны кривой. Вывод формулы для расчета кривизны кривой. Определение вектора бинормали в заданной точке кривой. Уравнения нормальной, спрямляющей и соприкасающейся плоскостей.
	\item Определения первообразной функции и неопределенного интеграла. Теорема о множестве первообразных заданной функции. Свойство линейности неопределенного интеграла. Методы замены переменной и интегрирования по частям \textit{(знать и уметь применять)}. 
\end{enumerate}



\end{document}